\documentclass[aspectratio=169]{beamer}
\usetheme{Compostela}



%%%%%%%%%%%%%%%%%%%%%%%%%%%%%%%%%%%%%%%%%%%%%%%%%%%%%%%%%%%%%%%%%%%%%%%%%%%%%%%%
%%%%%%%%%%%%%%%%%%%%%%%%%%%% Talk Configuration %%%%%%%%%%%%%%%%%%%%%%%%%%%%%%%%

\newcommand{\TalkPlace}{Santiago de Compostela}
\newcommand{\TalkAuthor}{Bernal de Bonaval}
\newcommand{\TalkAuthorShort}{Bernal de Bonaval}
\newcommand{\TalkTitle}{The great city of \\ \large {\fontspec{Zapfino} Santiago de Compostela}}
\newcommand{\TalkTitleShort}{The great city of Santiago de Compostela}
\newcommand{\TalkInstitute}{The Court of Alfonso X 'The Wise'}
\newcommand{\TalkDate}{February 30th}
\newcommand{\TalkDateNumber}{2020/02/30}

%%%%%%%%%%%%%%%%%%%%%%%%%%%%%%%%%%%%%%%%%%%%%%%%%%%%%%%%%%%%%%%%%%%%%%%%%%%%%%%%



\usepackage{array}

\usepackage{pgfplots}

\pgfplotsset{compat=1.10}
\usepgfplotslibrary{fillbetween}
\usetikzlibrary{patterns}
%\newcommand\hmmax{0}
%\newcommand\bmmax{0}














%%%%%%%%%%%%%%%%%%%%%%%%%%%%%%%%%%%%%%%%%%%%%%%%%%%%%%%%%%%%%%%%%%%%%%%%%%%%%%%%
\begin{document} %%%%%%%%%%%%%%%%%%%%%%%%%%%%%%%%%%%%%%%%%%%%%%%%%%%%%%%%%%%%%%%
%%%%%%%%%%%%%%%%%%%%%%%%%%%%%%%%%%%%%%%%%%%%%%%%%%%%%%%%%%%%%%%%%%%%%%%%%%%%%%%%



%%%%%%%%%%%%%%%%%%%%%%%%%%%%%%%%%%%%%%%%%%%%%%%%%%%%%%%%%%%%%%%%%%%%%%%%%%%%%%%%
%%%%%%%%%%%%%%%%%%%%%%%%%%%%% TITLE PAGE & CONTENTS %%%%%%%%%%%%%%%%%%%%%%%%%%%%
%%%%%%%%%%%%%%%%%%%%%%%%%%%%%%%%%%%%%%%%%%%%%%%%%%%%%%%%%%%%%%%%%%%%%%%%%%%%%%%%

\begin{frame}[plain,backgroundpicture=fig/scq_alameda,overlaytitlepage=0.9]
  % LOGOS GO HERE!
  \begin{minipage}[b][\textheight][b]{5cm}
    \includegraphics[height=1cm]{logos/alfonsoX_signum_reg}
  \end{minipage}
\end{frame}

\begin{frame}[plain,backgroundpicture=fig/scq_alameda,overlaytoc=0.9]
  \addtocounter{framenumber}{-1}
  \hspace*{7.3cm}\begin{minipage}{8cm}
    \tableofcontents
  \end{minipage}
\end{frame}

%%%%%%%%%%%%%%%%%%%%%%%%%%%%%%%%%%%%%%%%%%%%%%%%%%%%%%%%%%%%%%%%%%%%%%%%%%%%%%%%




\section{Introduction}

%%%%%%%%%%%%%%%%%%%%%%%%%%%%%%%%%%%%%%%%%%%%%%%%%%%%%%%%%%%%%%%%%%%%%%%%%%%%%%%%

\begin{frame}[backgroundpicture=fig/scq_map_street]
\frametitle{nosubsection}

\begin{columns}
  \begin{column}{0.34\textwidth}
    Placed in Northwestern Spain, Santiago de Compostela is the capital city of Galicia.
    The city has its origin in the shrine of Saint James the Great, now the Cathedral of Santiago de Compostela, as the destination of the Way of St. James, a leading Catholic pilgrimage route since the 9th century.\\[5mm]
    It has about 100 000 inhabitants spread over 220 km${}^2$
    and in 1985, the city's Old Town was designated a UNESCO World Heritage Site.
  \end{column}
  \hspace*{9.5cm}
\end{columns}

\end{frame}
%%%%%%%%%%%%%%%%%%%%%%%%%%%%%%%%%%%%%%%%%%%%%%%%%%%%%%%%%%%%%%%%%%%%%%%%%%%%%%%%












\section{Great city, awesome beamer theme}
\subsection{Motivation}


%%%%%%%%%%%%%%%%%%%%%%%%%%%%%%%%%%%%%%%%%%%%%%%%%%%%%%%%%%%%%%%%%%%%%%%%%%%%%%%%
\begin{frame}[default, allowframebreaks]
\frametitle{noframetitle}

Beamer themes are named after some well--known city, and I thought Santiago
deserves one among the best. And here it is!

\begin{itemize}
  \item This theme is built on top of basic latex \texttt{.sty} files that should be compiled with XeLaTeX.
  \item It should be quite easy to customize.
  \item Its main font is Fira Sans, so it must be installed beforehand.
  \item ...
\end{itemize}




Hey there!
\begin{block}{Quote}
Ever loved someone so much, you would do anything for them? Yeah, well make that someone yourself and do whatever the hell you want
\end{block}
\begin{block}{Quote}
Ever loved someone so much, you would do anything for them? Yeah, well make that someone yourself and do whatever the hell you want
\end{block}

\begin{enumerate}
  \item This is an item.
  \item Is this another one?
\end{enumerate}

\begin{itemize}
  \item This is an item.
  \item Is this another one?
\end{itemize}

\end{frame}
%%%%%%%%%%%%%%%%%%%%%%%%%%%%%%%%%%%%%%%%%%%%%%%%%%%%%%%%%%%%%%%%%%%%%%%%%%%%%%%%





\section{Delightful options}


\subsection{Options with backrounds}
% ------------------------------------------------------------------------------
\begin{frame}[backgroundpicture=fig/scq_berenguela,overlayfull=0.7]
\frametitle{noframetitle}
\framesubtitle{background + dimmed background}

\texttt{%
\textbackslash begin\{frame\}[backgroundpicture=fig/scq\_berenguela,overlayfull=0.7]\\%
\textbackslash frametitle\{noframetitle\}\\
\textbackslash framesubtitle\{background + dimmed background\}\\
\textbackslash end\{frame\}%
}

\end{frame}
% ------------------------------------------------------------------------------



% ------------------------------------------------------------------------------
\begin{frame}[backgroundpicture=fig/scq_berenguela,overlayslantedleft=0.7]
\frametitle{noframetitle}
\framesubtitle{background + semi dimmed background left}

\texttt{%
\textbackslash begin\{frame\}[backgroundpicture=fig/scq\_berenguela,overlayslantedleft=0.7]\\%
\textbackslash frametitle\{noframetitle\}\\
\textbackslash framesubtitle\{background + dimmed background\}\\
\textbackslash end\{frame\}%
}

\end{frame}
% ------------------------------------------------------------------------------



% ------------------------------------------------------------------------------
\begin{frame}[backgroundpicture=fig/scq_berenguela,overlayslantedright=0.7]
\frametitle{noframetitle}
\framesubtitle{background + semi dimmed background right}

\texttt{%
\textbackslash begin\{frame\}[backgroundpicture=fig/scq\_berenguela,overlayslantedright=0.7]\\%
\textbackslash frametitle\{noframetitle\}\\
\textbackslash framesubtitle\{background + dimmed background\}\\
\textbackslash end\{frame\}%
}

\end{frame}
% ------------------------------------------------------------------------------



% ------------------------------------------------------------------------------
\begin{frame}[plain, default, backgroundpicture=fig/scq_berenguela]
%\frametitle{noframetitle}
%\framesubtitle{background}


\end{frame}
% ------------------------------------------------------------------------------




\subsection{Options with frametitles}
% ------------------------------------------------------------------------------
\begin{frame}[default]
\frametitle{noframetitle}
\vspace*{-0.6cm}
\begin{columns}[t]
  \begin{column}{0.48\textwidth}
    \begin{block}{Section}
      \texttt{%
      \textbackslash begin\{frame\}\\%
      \textbackslash frametitle\{nosubsection\}\\
      CONTENTS\\%
      \textbackslash end\{frame\}%
      }\\
      like slide Introduction.
    \end{block}
    \begin{block}{Section + subsection + frametitle}
      \texttt{%
      \textbackslash begin\{frame\}\\%
      \textbackslash frametitle\{text\}\\
      CONTENTS\\%
      \textbackslash end\{frame\}%
      }\\
      like slide BLA.
    \end{block}
  \end{column}
  \begin{column}{0.48\textwidth}
    \begin{block}{Section + subsection}
      \texttt{%
      \textbackslash begin\{frame\}\\%
      \textbackslash frametitle\{noframetitle\}\\
      CONTENTS\\%
      \textbackslash end\{frame\}%
      }\\
      like this slide.
    \end{block}
    \begin{alertblock}{Section + subsection + frametitle + framesubtitle}
      \texttt{%
      \textbackslash begin\{frame\}\\%
      \textbackslash frametitle\{text\}\\
      \textbackslash framesubtitle\{text\}\\
      CONTENTS\\%
      \textbackslash end\{frame\}%
      }\\
      which I find quite overwhelming.
    \end{alertblock}
  \end{column}
\end{columns}

\end{frame}
% ------------------------------------------------------------------------------



\subsection{Predefined set of colors}
% ------------------------------------------------------------------------------
\begin{frame}[default]
\frametitle{noframetitle}

\centering
\textbf{Standard set of colors}\\[5mm]

\begin{tikzpicture}
  \draw[scqblue,   fill=scqblue,   thick] (0,0) circle (0.5);
  \draw[scqgreen,  fill=scqgreen,  thick] (1.5,0) circle (0.5);
  \draw[scqindigo, fill=scqindigo, thick] (3,0) circle (0.5);
  \draw[scqorange, fill=scqorange, thick] (4.5,0) circle (0.5);
  \draw[scqpink,   fill=scqpink,   thick] (6,0) circle (0.5);
  \draw[scqpurple, fill=scqpurple, thick] (7.5,0) circle (0.5);
  \draw[scqred,    fill=scqred,    thick] (9.0,0) circle (0.5);
  \draw[scqteal,   fill=scqteal,   thick] (10.5,0) circle (0.5);
  \draw[scqyellow, fill=scqyellow, thick] (12,0) circle (0.5);
\end{tikzpicture}

\tiny
\begin{tabular}{m{1.08cm}m{1.08cm}m{1.08cm}m{1.08cm}m{1.08cm}m{1.08cm}m{1.08cm}m{1.08cm}m{1.08cm}m{1.08cm}}
  \centering\texttt{scqblue} &
  \centering\texttt{scqgreen} &
  \centering\texttt{scqindigo} &
  \centering\texttt{scqorange} &
  \centering\texttt{scqpink} &
  \centering\texttt{scqpurple} &
  \centering\texttt{scqred} &
  \centering\texttt{scqteal} &
  \centering\texttt{scqyellow}
\end{tabular}
\normalsize

\vspace*{1cm}

\centering
\textbf{Darker set of colors}\\[5mm]

\begin{tikzpicture}
  \draw[scqblue,   fill=scqblue,   thick] (0,0) circle (0.5);
  \draw[scqgreen,  fill=scqgreen,  thick] (1.5,0) circle (0.5);
  \draw[scqindigo, fill=scqindigo, thick] (3,0) circle (0.5);
  \draw[scqorange, fill=scqorange, thick] (4.5,0) circle (0.5);
  \draw[scqpink,   fill=scqpink,   thick] (6,0) circle (0.5);
  \draw[scqpurple, fill=scqpurple, thick] (7.5,0) circle (0.5);
  \draw[scqred,    fill=scqred,    thick] (9.0,0) circle (0.5);
  \draw[scqteal,   fill=scqteal,   thick] (10.5,0) circle (0.5);
  \draw[scqyellow, fill=scqyellow, thick] (12,0) circle (0.5);
\end{tikzpicture}

\tiny
\begin{tabular}{m{1.08cm}m{1.08cm}m{1.08cm}m{1.08cm}m{1.08cm}m{1.08cm}m{1.08cm}m{1.08cm}m{1.08cm}m{1.08cm}}
  \centering\texttt{scqblue} &
  \centering\texttt{scqgreen} &
  \centering\texttt{scqindigo} &
  \centering\texttt{scqorange} &
  \centering\texttt{scqpink} &
  \centering\texttt{scqpurple} &
  \centering\texttt{scqred} &
  \centering\texttt{scqteal} &
  \centering\texttt{scqyellow}
\end{tabular}
\normalsize



\end{frame}
% ------------------------------------------------------------------------------


\subsection{Dark theme}


% ------------------------------------------------------------------------------
\begin{frame}[dark]
\frametitle{noframetitle}

\begin{center}
  \Large Goddammit! There is also \textbf{dark} theme,
  just writing \texttt{dark} as frame option.
\end{center}

\vspace{1cm}
\color{white}
\begin{enumerate}
  \item Dark theme is absolutely stunning, dramatic and easier on the
  eyes
  \item Dark theme is absolutely stunning, dramatic and easier on the
  eyes
\end{enumerate}

\end{frame}
% ------------------------------------------------------------------------------



% ------------------------------------------------------------------------------
\begin{frame}
\frametitle{noframetitle}

\centering
\textbf{Standard set of colors}\\[5mm]

\begin{tikzpicture}
  \draw[scqblueopt,   fill=scqblueopt,   thick] (0,0) circle (0.5);
  \draw[scqgreenopt,  fill=scqgreenopt,  thick] (1.5,0) circle (0.5);
  \draw[scqindigoopt, fill=scqindigoopt, thick] (3,0) circle (0.5);
  \draw[scqorangeopt, fill=scqorangeopt, thick] (4.5,0) circle (0.5);
  \draw[scqpinkopt,   fill=scqpinkopt,   thick] (6,0) circle (0.5);
  \draw[scqpurpleopt, fill=scqpurpleopt, thick] (7.5,0) circle (0.5);
  \draw[scqredopt,    fill=scqredopt,    thick] (9.0,0) circle (0.5);
  \draw[scqtealopt,   fill=scqtealopt,   thick] (10.5,0) circle (0.5);
  \draw[scqyellowopt, fill=scqyellowopt, thick] (12,0) circle (0.5);
\end{tikzpicture}

\tiny
\begin{tabular}{m{1.08cm}m{1.08cm}m{1.08cm}m{1.08cm}m{1.08cm}m{1.08cm}m{1.08cm}m{1.08cm}m{1.08cm}m{1.08cm}}
  \centering\texttt{scqblue} &
  \centering\texttt{scqgreen} &
  \centering\texttt{scqindigo} &
  \centering\texttt{scqorange} &
  \centering\texttt{scqpink} &
  \centering\texttt{scqpurple} &
  \centering\texttt{scqred} &
  \centering\texttt{scqteal} &
  \centering\texttt{scqyellow}
\end{tabular}
\normalsize

\vspace*{1cm}

\centering
\textbf{Darker set of colors}\\[5mm]

\begin{tikzpicture}
  \draw[scqblue,   fill=scqblue,   thick] (0,0) circle (0.5);
  \draw[scqgreen,  fill=scqgreen,  thick] (1.5,0) circle (0.5);
  \draw[scqindigo, fill=scqindigo, thick] (3,0) circle (0.5);
  \draw[scqorange, fill=scqorange, thick] (4.5,0) circle (0.5);
  \draw[scqpink,   fill=scqpink,   thick] (6,0) circle (0.5);
  \draw[scqpurple, fill=scqpurple, thick] (7.5,0) circle (0.5);
  \draw[scqred,    fill=scqred,    thick] (9.0,0) circle (0.5);
  \draw[scqteal,   fill=scqteal,   thick] (10.5,0) circle (0.5);
  \draw[scqyellow, fill=scqyellow, thick] (12,0) circle (0.5);
\end{tikzpicture}

\tiny
\begin{tabular}{m{1.08cm}m{1.08cm}m{1.08cm}m{1.08cm}m{1.08cm}m{1.08cm}m{1.08cm}m{1.08cm}m{1.08cm}m{1.08cm}}
  \centering\texttt{scqblue} &
  \centering\texttt{scqgreen} &
  \centering\texttt{scqindigo} &
  \centering\texttt{scqorange} &
  \centering\texttt{scqpink} &
  \centering\texttt{scqpurple} &
  \centering\texttt{scqred} &
  \centering\texttt{scqteal} &
  \centering\texttt{scqyellow}
\end{tabular}
\normalsize



\end{frame}
% ------------------------------------------------------------------------------



%%%%%%%%%%%%%%%%%%%%%%%%%%%%%%%%%%%%%%%%%%%%%%%%%%%%%%%%%%%%%%%%%%%%%%%%%%%%%%%%
%%%%%%%%%%%%%%%%%%%%%%%%%%%%% TITLE PAGE & CONTENTS %%%%%%%%%%%%%%%%%%%%%%%%%%%%
%%%%%%%%%%%%%%%%%%%%%%%%%%%%%%%%%%%%%%%%%%%%%%%%%%%%%%%%%%%%%%%%%%%%%%%%%%%%%%%%

\begin{frame}[plain,dark,backgroundpicture=fig/scq_alameda,overlaytitlepage=0.9]
  % LOGOS GO HERE!
  \begin{minipage}[b][\textheight][b]{5cm}
    \includegraphics[height=1cm]{logos/alfonsoX_signum_reg}
  \end{minipage}
\end{frame}

\addtocounter{framenumber}{+1}
\begin{frame}[plain,dark,backgroundpicture=fig/scq_alameda,overlaytoc=0.9]
  \addtocounter{framenumber}{-1}
  \hspace*{7.3cm}\begin{minipage}{8cm}
    \tableofcontents
  \end{minipage}
\end{frame}

%%%%%%%%%%%%%%%%%%%%%%%%%%%%%%%%%%%%%%%%%%%%%%%%%%%%%%%%%%%%%%%%%%%%%%%%%%%%%%%%




% ------------------------------------------------------------------------------
\begin{frame}[default,dark]
\frametitle{noframetitle}

\Large \centering Goddammit! There is also \textbf{dark} theme, just writing \texttt{dark}
as frame option.

\end{frame}
% ------------------------------------------------------------------------------




% ------------------------------------------------------------------------------
\begin{frame}[default,plain,dark]
\frametitle{noframetitle}

\begin{tikzpicture}[scale=0.5]
   \node[anchor=south west,inner sep=0] at (0,0) {\includegraphics[width=5cm]{fig/scq_galicia_museum}};

  \draw [cyan, xshift=4.8cm, yshift=4.8cm, name path=one] plot [smooth, tension=1] coordinates { (1.36128, 0.543927) (1.347, 0.680756) (1.2817, 0.865099) (1.16907,
     1.0763) (1.01334, 1.29655) (0.819248, 1.51071) (0.591981,
    1.70617) (0.337076, 1.87272) (0.060372, 2.00235) (-0.232059,
    2.08915) (-0.53398, 2.12915) (-0.839055, 2.12015) (-1.14092,
    2.06157) (-1.43324, 1.95431) (-1.70978, 1.80062) (-1.9645,
    1.60391) (-2.19155, 1.36862) (-2.38542, 1.10007) (-2.54095,
    0.804307) (-2.65343, 0.487956) (-2.71864,
    0.158069) (-2.73293, -0.17803) (-2.6933, -0.512887) (-2.59744,
  -0.839074) (-2.44381, -1.14934) (-2.23172, -1.43675) (-1.96137,
  -1.69487) (-1.63393, -1.91785) (-1.25162, -2.10066) (-0.81775,
  -2.23916) (-0.336808, -2.33033) (0.185481, -2.37232) (0.742089,
  -2.36472) (1.32462, -2.30859) (1.92322, -2.20672) (2.52657,
  -2.0637) (3.12174, -1.8861) (3.69418, -1.68262) (4.22764,
  -1.46426) (4.70409, -1.24443) (5.10365, -1.03912) };

  \draw [cyan, xshift=4.8cm, yshift=4.8cm, name path=two] plot [smooth, tension=1] coordinates {(5.12062,0.5693) (4.98449,0.662438) (4.81846,0.579129) (4.61267,0.369576) (4.36091,0.0773492) (4.0603,-0.260239) (3.71089,-0.611766) (3.3154,-0.951333) (2.87884,-1.25819) (2.40821,-1.51637) (1.91216,-1.71433) (1.40064,-1.84454) (0.88459,-1.90317) (0.375611,-1.8897) (-0.114389,-1.80651) (-0.57351,-1.65858) (-0.990194,-1.45307) (-1.35356,-1.19897) (-1.65373,-0.906726) (-1.88216,-0.587864) (-2.03199,-0.254636) (-2.09835,0.0803597) (-2.07871,0.404546) (-1.97319,0.705737) (-1.78492,0.972508) (-1.52035,1.19456) (-1.18959,1.36311) (-0.806752,1.47121) (-0.39025,1.51419) (0.0368356,1.48995) (0.446441,1.3994) (0.805187,1.24679) (1.07404,1.04006) (1.20801,0.79126) (1.15575,0.516898) (0.859317,0.238296)};
   % \draw [cyan, xshift=2.8cm, yshift=2.5cm, domain=2:10, variable=\t, smooth, samples=75, name path=one] plot [smooth, tension=3] ({0.5*\t r}: {0.0+0.05*\t+0.065*\t*\t-0.002*\t*\t*\t-0.00025*\t*\t*\t*\t-0.000003*\t*\t*\t*\t*\t});


   %\draw [thin, dashed,draw=gray!50!blue!30] (0,0) grid (3,3);
   \draw (0,3)--(0,0)node[below]{$0$} --(3,0)node[below]{$1$}--(3,3);
    \tikzfillbetween[
      of=one and two,split
    ] {green};
\end{tikzpicture}

\end{frame}
% ------------------------------------------------------------------------------

%{0.05+0.01*\t+0.12*\t*\t-0.011*\t*\t*\t});



%%%%%%%%%%%%%%%%%%%%%%%%%%%%%%%%%%%%%%%%%%%%%%%%%%%%%%%%%%%%%%%%%%%%%%%%%%%%%%%%
\end{document} %%%%%%%%%%%%%%%%%%%%%%%%%%%%%%%%%%%%%%%%%%%%%%%%%%%%%%%%%%%%%%%%%
%%%%%%%%%%%%%%%%%%%%%%%%%%%%%%%%%%%%%%%%%%%%%%%%%%%%%%%%%%%%%%%%%%%%%%%%%%%%%%%%
